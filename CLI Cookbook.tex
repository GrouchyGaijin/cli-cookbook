\documentclass[12pt,a4paper]{article}
\usepackage[utf8]{inputenc}
\usepackage{fancyvrb}
\usepackage{color}
\usepackage[linktoc=all]{hyperref}
\hypersetup{
    colorlinks,
    citecolor=black,
    filecolor=black,
    linkcolor=black,
    urlcolor=black
}

\begin{document}
\title{CLI Cookbook}
\author{Lucas Westermann and readers of Command \& Conquer}
\maketitle
\textbf{Note: Each command in a longer list is separated by a blank line.  All commands on more than one line without a blank line are simply wrapped text - run it as one command! } \\
You can find this on GitHub: \url{https://github.com/lswest/cli-cookbook}\\
If you want to contribute - feel free to clone the repository, or open an issue (if you aren't comfortable with LaTeX)
\pagebreak
\tableofcontents
\pagebreak

\newcommand\codeHighlight[1]{\textcolor[rgb]{1,0,0}{\textbf{#1}}}

\section{Get information on commands}
\label{Get information on commands}

\subsection{Get a brief summary of bash commands}
\begin{Verbatim}[commandchars=\\\{\}]
\codeHighlight{whatis} command
\end{Verbatim}
Replace command with the name of the command you're curious about, and whatis should return a brief summary of what it does (I say should, as not all programs register themselves in the whatis database).

\subsection{Get the location of a command's binary}
\begin{Verbatim}[commandchars=\\\{\}]
\codeHighlight{whereis} command

\codeHighlight{which} command
\end{Verbatim}
Whereis and which both return the path to the main executable of the command.

\section{Finding and managing files}
\label{Finding and managing files}
\subsection{sed}  
\begin{Verbatim}[commandchars=\\\{\}]
\codeHighlight{sed}
Example: sed ‘s/[pP][iI][nN][kK] [fF][lL][oO][yY][dD]/Pink Floyd/g’
\end{Verbatim}
Sed can be used for word replacement in files.  The example supplied will consolidate all spelling variations of Pink Floyd.  It's also frequently used to unify file types (i.e. .JPG .jpeg .jpg into one uniform extension).

\subsection{ls}
\begin{Verbatim}[commandchars=\\\{\}]
\codeHighlight{ls -a|grep} \$searchterm
Note: Replace \$searchterm with the term you're looking for.
\end{Verbatim}
This combination has ls list all files (including hidden files and directories), and passes the resulting list to grep, which then filters it according to your search term.  The manpage of grep will give you the syntax to negate searches ("find everything that doesn't contain $X$").

\subsection{convert command}
\begin{Verbatim}[commandchars=\\\{\}]
\codeHighlight{convert}
Example: convert *.jpg output.pdf
\end{Verbatim}
Convert is a powerful command-line tool that helps to convert between formats.  However, it can also be used to merge files together.  The example above takes all files that end in .jpg and merges them into a single PDF.  The reader who supplied this notes that they find it most useful when scanning large numbers of files.

\subsection{Search with updatedb and locate}
\label{locate}
\begin{Verbatim}[commandchars=\\\{\}]
\codeHighlight{sudo updatedb} # updates database

\codeHighlight{locate} search term
\end{Verbatim}
The locate command quickly searches the database for items matching the search terms supplied.  It won't always find the file you're looking for, but is faster than find if you don't know what folder it's in.

\subsection{Search with find}
\begin{Verbatim}[commandchars=\\\{\}]
\codeHighlight{find }\$path\codeHighlight{ | grep -Ri "}search term\codeHighlight{"}

\codeHighlight{find }\$path\codeHighlight{  -name "}search term\codeHighlight{"}

\codeHighlight{find }\$path\codeHighlight{ -iname "}search term\codeHighlight{"}
\end{Verbatim}
Replace \textbf{search term} with your search terms.  Find will also search from the specified path onwards (for current path, just use '\textbf{.}' minus the quotes).  I wouldn't recommend using find to search through your directory from root ('/'), as it could take forever.  Much better to use locate there.  The only difference between -name and -iname is that -iname is case insensitive, which means .JPG and .jpg are treated as the same text.

\subsection{Search every file recursively}
\begin{Verbatim}[commandchars=\\\{\}]
\codeHighlight{find . -type f |xargs grep '}'text to search\codeHighlight{'}
\end{Verbatim}
The find command searches all files (-type f) in the current directory and below, using the search terms supplied to grep and parsed through xargs.

\subsection{Convert files using LibreOffice}
\begin{Verbatim}[commandchars=\\\{\}]
\codeHighlight{soffice --headless --nologo --convert-to odp *.ppt}
# set source and destination formats as desire
\end{Verbatim}
Similar to the convert command, this can convert between file formats (any formats you can use/create in LibreOffice).  The first two arguments are to hide the LibreOffice interface, and the third argument tells it what to do.  The reader who supplied this command mentioned that they found it useful for converting the PowerPoint slides they received from their professors to PDF or ODP.  However, converting the latest Microsoft formats (.docx, .pptx, etc.) can/will result in formatting issues in LibreOffice 3.5 in Ubuntu 12.04.  It may also impact later versions, but this was the warning supplied with the command.

\section{Packages}
\label{Packages}
\subsection{Debian based systems}
All the following commands are for Debian based systems (which is what Ubuntu is).  In the case of trusted keys - not all Debian systems have them. 
\subsubsection{Remove leftover configuration files from removed packages}
\begin{Verbatim}[commandchars=\\\{\}]
\codeHighlight{sudo dpkg --purge `dpkg -l | grep ^rc | awk '{print \$2}'`}
\end{Verbatim}


\subsubsection{Create a detailed list of installed packages}
\begin{Verbatim}[commandchars=\\\{\}]
\codeHighlight{COLUMNS=200 dpkg-query -l > packages_list.list}
\end{Verbatim}
COLUMNS is a variable being passed to dpkg-query, and isn't a mandatory part of the command.  In this case, however, it helps to format the results for the resulting text file.  It essentially queries dpkg for a list of installed packages.  The resulting list will contain package names, versions and descriptions. Good for looking up what’s installed and what is done by which package. Alphabetical order.

\subsubsection{Create a list of installed packages for restoring your system}
\begin{Verbatim}[commandchars=\\\{\}]
\codeHighlight{dpkg --get-selections | awk '!/deinstall|purge|hold/ {print \$1}' > packages.list}
\end{Verbatim}
This list only contains only package names of installed packages. One per line. It is therefore suitable for re-installing all packages from the command line (see below). Alphabetical order.

\subsubsection{Store package state in a file}
\begin{Verbatim}[commandchars=\\\{\}]
\codeHighlight{apt-mark showauto > package-states-auto}
\end{Verbatim}
This command generates a list containing the state of each installed package, and saves it to the file package-states-auto.  This is needed for restoring an installation properly.

\subsubsection{Restore a list of packages}
\begin{Verbatim}[commandchars=\\\{\}]
\codeHighlight{dpkg --get-selections | awk '!/deinstall|purge|hold/ {print \$1}' > packages.list}
\end{Verbatim}

\subsubsection{Save all software sources to one file}
\begin{Verbatim}[commandchars=\\\{\}]
\codeHighlight{find /etc/apt/sources.list -type f -name '*.list' -exec bash -c 'echo }
\codeHighlight{-e "\textbackslash n## \$1 ";grep "^[[:space:]]*[^#[:space:]]" \${1}' _ \{\} \textbackslash; }
\codeHighlight{> sources.list.save}
\end{Verbatim}
This long command finds all entries in sources.list and the sub-directories, putting the entire list in a file called sources.list.save.

\subsubsection{Save the keys of trusted software sources}
\begin{Verbatim}[commandchars=\\\{\}]
\codeHighlight{sudo cp /etc/apt/trusted.gpg trusted-keys.gpg}
\end{Verbatim}
This command simply makes a copy of the trusted key list (stored under /etc/apt/) and copies it to the current directory.

\subsubsection{Restore installation}

\begin{Verbatim}[commandchars=\\\{\}]
# firstly, restore sources manually!
\codeHighlight{sudo apt-key add trusted-keys.gpg}  # import keyring

\codeHighlight{sudo apt-get update} # update sources

\codeHighlight{xargs -a "packages.list" sudo apt-get install}  # install software

\codeHighlight{xargs -a "package-states-auto" sudo apt-mark auto} # set package state
\end{Verbatim}
The way I would recommend restoring the software sources is to copy and paste sources not already present in the existing sources.list file.  It can be tedious, but can avoid issues with different versions and installations.  The reader who supplied this said that they apply the above procedure to save time when re-installing a system. Or when they have to create a few identical installations.

\subsection{Archlinux}

\subsubsection{Create list of installed packages}
\label{Arch-installed-packages}
\begin{Verbatim}[commandchars=\\\{\}]
\codeHighlight{pacman -Qqe | grep -v "\$(pacman -Qqm)" > pkglist-off.txt}

\codeHighlight{pacman -Qqm > pkglist-loc.txt}
\end{Verbatim}
The first command creates a list of all officially installed packages (i.e. nothing from the AUR).  The second command creates a list of all locally installed packages.  If you use an AUR manager, copy the install package over along with the backup list - or else make note of the URL so you can wget it later.

\subsubsection{Create a local copy of installed packages}
\begin{Verbatim}[commandchars=\\\{\}]
\codeHighlight{cp /var/cache/pacman/pkg/*} \$backup-location

\codeHighlight{sudo pacman -Sc}
\end{Verbatim}
The first command will copy the local cached copies of installed packages.  However, pacman stores old copies too, so the second command will remove all the old copies, in case space is an issue.  Most AUR managers offer a similar option of caching files (not necessarily enabled by default).

\subsubsection{Restoring package installations}
\begin{enumerate}
\item Install Arch as you would usually from a live CD using the AIF (Arch Installation Framework).
\item Mount backup device (in live CD)
\begin{Verbatim}[commandchars=\\\{\}]
\codeHighlight{mkdir /backup-files}

\codeHighlight{mount }/dev/<disk-drive-partition>

\codeHighlight{/backup-files}
\end{Verbatim}
\item Now copy these to the newly created Arch install (mounted under /mnt)
\begin{Verbatim}[commandchars=\\\{\}]
\codeHighlight{mkdir -p /mnt/opt/restore}

\codeHighlight{cd /backup-files}

\codeHighlight{cp -a * /mnt/opt/restore}
\end{Verbatim}
\item Now chroot (change root) to the new Arch install
\begin{Verbatim}[commandchars=\\\{\}]
\codeHighlight{cd /mnt}

\codeHighlight{cp /etc/resolv.conf /mnt/etc}

\codeHighlight{mount -t proc none /mnt/arch/proc}

\codeHighlight{mount -t sysfs none /mnt/arch/sys}

\codeHighlight{mount -o bind /dev /mnt/arch/dev}

\codeHighlight{chroot . /bin/bash}
\end{Verbatim}
\item OPTIONAL: If you made a local copy of the cache, copy the cache:
\begin{Verbatim}[commandchars=\\\{\}]
\codeHighlight{cp /mnt/opt/restore/pkg/ /var/cache/pacman/pkg/}
\end{Verbatim}
\item Now install the official packages into the new system
\begin{Verbatim}[commandchars=\\\{\}]
\codeHighlight{pacman -Sy}

\codeHighlight{pacman -S --needed \$(cat /opt/restore/pkglist-off.txt)}
\end{Verbatim}
\item If you have AUR packages installed, install the AUR helper, and then do the following
\begin{Verbatim}[commandchars=\\\{\}]
\$AURhelper \codeHighlight{-S \$(cat /opt/restore/pkglist-loc.txt | grep -vx }
\codeHighlight{"\$(pacman -Qqm)")}
# Replace \$AURhelper with the name of your preferred helper
\end{Verbatim}
The above command installs all packages listed in the local files backup from \hyperref[Arch-installed-packages]{\ref*{Arch-installed-packages} }.
\end{enumerate}
This long process will result in having your old packages installed into a new system.  This can help make the occasional re-install infinitely easier.  Or, if you have a system you're completely satisfied with, you can then move it to a new system.  The ArchWiki has plenty of information for those of you trying to move more than just the installed packages.

\section{Hard drive commands}
\label{Hard drive commands}
\subsection{Full list of partitions}
\begin{Verbatim}[commandchars=\\\{\}]
\codeHighlight{sudo fdisk -l} # for legacy BIOS

\codeHighlight{sudo gdisk -l} # for GPT (UEFI) systems
\end{Verbatim}
These commands will print a list of all hard drives and their partitioning structure.  If you're unsure of which command to use, fdisk is probably for you.  Don't worry - no harm running the wrong command, and it will prompt you that you ought to be using the other command.

\subsection{Back up MBR (Master Boot Record)}
\begin{Verbatim}[commandchars=\\\{\}]
\codeHighlight{sudo dd if=/dev/hda of=mbr_backup bs=512 count=1}
\end{Verbatim}
This command runs dd on the "input file" (if) of /dev/hda (replace this with the disk you want to back up the MBR of \- most likely /dev/sda).  It then reads the first 512 bits (bs=512 \- size count=1 - how many chunks of size 512 to read) and saves them in the "output file" (of) mbr\textunderscore backup.  If you're running this from a live CD or disk, remember to save/move the output file to a persistent drive (i.e. hard drive, USB stick, etc.).  Otherwise it will be lost the moment you reboot.

\subsection{Back up boot code}
\begin{Verbatim}[commandchars=\\\{\}]
\codeHighlight{dd if=/dev/hda of=bootcode_backup bs=446 count=1}
\end{Verbatim}
Similar as with the MBR command, this command backs up a section of the hard drive.  In this case, it backs up only the section of the hard drive that contains boot code.

\subsection{Display detailed information on your Hard Disk}
\begin{Verbatim}[commandchars=\\\{\}]
\codeHighlight{sudo hdparm -I} /dev/sdx
\end{Verbatim}
This command lists thorough information about the supplied hard disk (replace /dev/sdx with your actual drive).

\subsection{Correct partitioning order}
According to the reader who submitted this:
"\textit{Due to using gparted over and over again for my external 2Gb harddisk my partitioning was off (first p2, then p3, p1 was the extended, in there p7, then p5, then p6).
I corrected this by doing:}"
\begin{Verbatim}[commandchars=\\\{\}]
\codeHighlight{sudo sfdisk -d }/dev/sdb\codeHighlight{ > }sdb.old

\codeHighlight{sudo cp }sdb.old sdb.new

\codeHighlight{sudo gedit} sdb.new 

\# Change order, don’t forget p4 which did not exist
\codeHighlight{sudo sfdisk }/dev/sdb\codeHighlight{ < }sdb.new

If it's wrong, do the following:
\codeHighlight{sudo sfdisk }/dev/sdb\codeHighlight{ < }sdb.old
Then retry the gedit step.
\end{Verbatim}
I'm honestly of two minds about this suggestion.  On one hand, I can accept doing this on usb sticks - though I don't really think you'll need more than one partition on any but the largest.  Doing this on any kind of actual hard drive seems extremely dangerous as countless things could go wrong.  An out of sequence partition table isn't a problem in any way, so only risk this \underline{\textbf{\textit{if you can't possibly live with it}}}

\subsection{Fill hard disk with zeros}
\begin{Verbatim}[commandchars=\\\{\}]
\textbf{BE CAREFUL!!!}
\codeHighlight{dd if=/dev/zero of=}/dev/sdb \codeHighlight{bs=64k}
\textbf{BE CAREFUL!!!}
\end{Verbatim}
This command can be extremely dangerous to use if you're not sure what you're doing.  It writes to the disk and fills it with zeros, effectively wiping the hard drive.  For the security-conscious of you, doing this a couple of times on a drive before throwing it away/giving it away, it will make it much more difficult to restore data from the drive.  It's best used in conjunction with the following command.  Be sure to replace /sda/sdb with the correct path.

\subsection{Fill hard drive with random data}
\begin{Verbatim}[commandchars=\\\{\}]
\textbf{BE CAREFUL!!!}
\codeHighlight{dd if=/dev/urandom of=}/dev/sdb
\textbf{BE CAREFUL!!!}
\end{Verbatim}
This command is extremely dangerous to use blindly without understanding what you're doing.  This is because, similar to the previous command, it fills the hard drive with data.  Instead of zeros, it instead uses random data.  The reader who submitted this code mentions as well that they do this with old drives before selling them.  They also mention that their preference is to re-format them as \textbf{NTFS} at the end as well.  Remember to replace /dev/sdb with the correct path.

\section{Password generation}
\label{Password generation}
\subsection{Numeric passwords}
\begin{Verbatim}[commandchars=\\\{\}]
\codeHighlight{makepasswd --string 0123456789 --chars 32}
\end{Verbatim}
This command generates a single 32-character long numeric password.

\subsection{Alphanumeric passwords with special characters}
\begin{Verbatim}[commandchars=\\\{\}]
\codeHighlight{pwgen -sy 32 64}
\end{Verbatim}
This command generates 64 passwords of a 32-character length that contains letters, numbers and special characters.  The reason why we show how to generate multiple ones at first is because it can be useful when setting up a large number of networks at once.  \textbf{pwgen is not installed by default}

\section{wget}
\subsection{Make an offline copy of a website}
\label{}Make an offline copy of a website
\begin{Verbatim}[commandchars=\\\{\}]
\codeHighlight{wget --user-agent="Mozilla/5.0 (X11; U; Linux i686; de; rv:1.9b5)}
\codeHighlight{Gecko/2008050509 Firefox/3.0b5" -r -k -E -l 8} http://example.com
\end{Verbatim}
This command uses wget to impersonate Firefox's user agent, and then runs then runs recursively through the website (-r defaults to 5 levels, and -l sets the depth, to 8 in this case).  The -k switch tells wget to fix the links to make them suitable for local viewing, -E tells it to adjust extensions (i.e. change .asp to .html).  Don't forget to change the URL!

\subsection{Download a file (with the ability to continue)}
\begin{Verbatim}[commandchars=\\\{\}]
\codeHighlight{wget -c }http://example.com
\end{Verbatim}
wget grabs the supplied link and downloads the file - in the example it would download the index.html file, and if you give it a file link, it will download the file.  The -c switch lets you continue a download if it's interrupted.

\section{Display hardware information}
\label{Display hardware information}
\begin{Verbatim}[commandchars=\\\{\}]
\codeHighlight{lshw -C} SECTION
Example: sudo lshw -C network # lists all information about networking devices
Formatting example (HTML): sudo lshw -html > hardware.html #creates an html 
page with the lshw results

\codeHighlight{lsusb}
Example: sudo lsusb -v | more # lists verbose output and drops it into the more 
tool for easy reading.

\codeHighlight{lspci -nn}
# -nn prints also device and vendor id. useful if you’re looking for a driver, 
e.g. for WLAN, using databases at ubuntuusers.de or other

\codeHighlight{dmesg | grep KEYWORD}

\codeHighlight{ifconfig}

\codeHighlight{iwconfig}

\codeHighlight{dmidecode -t TYPENUMBER}
\end{Verbatim}
Logically, lsusb prints information about USB devices, and lspci lists information about PCI devices (system bus).  Lshw lists information about almost all hardware in your computer.  Dmesg prints out system messages, ifconfig prints hardware information about ethernet and wireless interfaces, and iwconfig display information on wireless information (such as ESSID).  Dmidecode dumps a computer's DMI (sometimes called SMBIOS) table contents into a human-readable format.  SMBIOS stores specific information about the computer.

The name for the SECTION (called class by lshw) can be found by running either of the following commands:
\begin{Verbatim}[commandchars=\\\{\}]
\codeHighlight{lshw -businfo}

\codeHighlight{lshw -short}
\end{Verbatim}

\section{View running processes}
\label{View running processes}
\subsection{htop}
\begin{Verbatim}[commandchars=\\\{\}]
\codeHighlight{htop}
\end{Verbatim}
Htop is a more functional (I find) version of top.  It's not always installed by default, but I have yet to find an official repository that doesn't offer the package.

\subsection{View logged in users and processes they own}
\begin{Verbatim}[commandchars=\\\{\}]
\codeHighlight{w}
\end{Verbatim}
The command "w" lists all logged in users and the processes associated with those accounts.

\section{Administration}
\label{Administration}

\subsection{Add user to group}
\begin{Verbatim}[commandchars=\\\{\}]
\codeHighlight{usermod -aG ‘}groupname\codeHighlight{’ ‘}user\codeHighlight{’}
\end{Verbatim}
This command modifies a user and adds them to the group specific as groupname.

\subsection{Create aliases in your shell}
\label{aliases}
This applies to almost any shell, though the configuration file will be different.  I will assume the use of the Bourne Again Shell (Bash) and therefore the file .bashrc.
\begin{Verbatim}[commandchars=\\\{\}]
\codeHighlight{vim ~/.bashrc}

Then add this anywhere in the bashrc file:
\codeHighlight{alias }\$name\codeHighlight{="}command\codeHighlight{"}

This alias will then be active in any new shell you open.
If you want to use it in a currently open shell, run this:
\codeHighlight{source ~/.bashrc}
\end{Verbatim}
After that, just type whatever the name of your alias was, and it will run the specified command.  Remember to replace \$name and command with the actual name you'd like the alias to have, and the command you want it to run.

\subsection{Check who opened which ports}
\begin{Verbatim}[commandchars=\\\{\}]
\codeHighlight{sudo netstat -lntup}
\end{Verbatim}
This is a fairly straightforward command - it lists users and the ports the have opened.

\section{Miscellaneous fixes}
\subsection{Upside down webcam in Skype}
Before using this command, you should find the path of your v4l1compat.so file.  For information on how to find it, see \hyperref[locate]{\ref*{locate}} (using updatedb and locate)
\begin{Verbatim}[commandchars=\\\{\}]
\codeHighlight{bash -c 'LD_PRELOAD=}/usr/lib/i386-linux-gnu/libv4l/v4l1compat.so \codeHighlight{skype'}

#if you're on 11.10, you'll need the following:
\codeHighlight{sh -c 'LD_PRELOAD=}/usr/lib/i386-linux-gnu/libv4l/v4l1compat.so 
\codeHighlight{/usr/bin/skype "\$@"'}
\end{Verbatim}
Since you most likely don't want to type this every time, you can alias it in your bashrc file (\hyperref[aliases]{\ref*{aliases}}), or else create a small bash script executable.

\subsection{GRUB boots into black screen}
Occasionally some graphics cards have issues booting into GRUB's framebuffer, which results in a black screen.  If you run into this issue, you can try adding \textbf{nomodeset} to the end of your kernel boot line.  GRUB2 and legacy GRUB have different ways to edit entries, but the bottom of the GRUB menu should have instructions.

\section{Video tips}
\subsection{ffmpeg}
\subsubsection{Record your desktop}
If you don't mind using a pre-build tool, using recordmydesktop is probably easiest.  However, if you want total control over the recording, you can use ffmpeg
\begin{Verbatim}[commandchars=\\\{\}]
\codeHighlight{ffmpeg -f alsa -ac 2 -i hw:0,0 -f x11grab -r 20 -s 1680x1050+0+0 -i :0.0 }
\codeHighlight{-acodec pcm_s16le -vcodec libx264 -vpre lossless_ultrafast -threads 0 -y }
\codeHighlight{vidcap.mkv}
\end{Verbatim}

\subsubsection{Use ffmpeg to cut a section of video out}
\begin{Verbatim}[commandchars=\\\{\}]
\codeHighlight{ffmpeg -ss} 00:02:23\codeHighlight{ -i} input.foo\codeHighlight{ -t }25\codeHighlight{ -vcodec copy -acodec copy} output.foo
\end{Verbatim}
This command tells ffmpeg that starting from 2 minutes and 23 seconds into the video, to copy out the following 25 seconds (-t 25) from input.foo into output.foo

\subsection{Use vdpau (hardware accelerated decoding in mplayer}
Nvidia uses the vdpau library to offload video encoding and decoding to the graphics card.  To use this in mplayer, do the following:
\begin{Verbatim}[commandchars=\\\{\}]
\codeHighlight{mplayer -vo vdpau -vc ffh264vdpau }file
\end{Verbatim}
Replace the word "file" with the path to the video you want to play.

\section{Useful Links}
\begin{itemize}
\item{\url{http://www.linuxcommand.org}}
\item{\url{http://www.commandlinefu.com}}
\item{\url{http://www.shell-fu.org}}
\item{\url{http://www.ss64.com}}
\end{itemize}
\begin{Verbatim}[commandchars=\\\{\}]
\codeHighlight{}
\end{Verbatim}
\end{document}
